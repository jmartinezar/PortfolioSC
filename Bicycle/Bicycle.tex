\documentclass{article}
\usepackage[left=2.54cm, right=2.54cm, bottom=2.54cm, top=2.54cm]{geometry}
\usepackage{amsmath}
\usepackage{graphicx}

\title{Bicycle movement}
\author{Juan Sebasti\'{a}n Mart\'{i}nez Ar\'{e}valo}

\begin{document}
\maketitle

\begin{abstract}
This document contains information about the project to model the movement of a bicycle. It briefly explains the theory of movement and the computational model used to obtain numerical results.
\end{abstract}

\section{First approach.}

The first model to consider is one in which the bicycle is frictionless. The equation that describe the movement of a body in classical mechanics is Newton's second law, which in one dimension is describes by:

\begin{equation}
  F = m\frac{dv}{dt}
\end{equation}

clearing $\frac{dv}{dt}$

\begin{equation}
  \frac{dv}{dt} = F/m
  \label{NSL}
\end{equation}

with this, we can determine the velocity of the bicycle at any time. However,  the force $F$ isn't easy to determine, so it's more convenient to perform the analysis in terms of other parameter like the power $P$, which represents the energy given to the physical system per unit of time. To do this, consider the mathematic definition of power, which is

\begin{equation}
  P = \frac{dE}{dt}
\end{equation}

since the energy in the system is purely kinetic, it takes the form $E=\frac{1}{2}mv^{2}$. Therefore the derivative is $dE/dt = mv(dv/dt)$. Clearing it

\begin{equation}
  \frac{dv}{dt} = \frac{P}{mv}
\end{equation}

integring, take the velocity at initial time $t_0$ as $v_0$ gives like result

\begin{equation}
  v(t) = \sqrt{v_0^2 + \frac{P}{m}t}
\end{equation}

However, this result is unphysical in the sense that the velocity increases indefinitely. Therefore, we consider the derivative of velocity as a finite difference


$$ \frac{v_{i+1} - v_i}{\Delta t} = \frac{P}{mv_{i}} \rightarrow v_{i+1} = v_i + \frac{P}{mv_{i}}\Delta t$$

Therefore, given an initial velocity, it is possible to obtain the velocity for any future time. A graph of velocity versus time is shown in the Figure \ref{f1}.

\begin{figure}[!h]
  \centering
  \includegraphics[width=0.7\linewidth]{velocity_v1.pdf}
  \caption{}
  \label{f1}
\end{figure}

\begin{figure}[!h]
  \centering
  \includegraphics[width=0.7\linewidth]{velocity_v2.pdf}
\end{figure}

\end{document}
