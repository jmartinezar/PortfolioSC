\documentclass{article}
\usepackage[left=2.54cm, right=2.54cm, bottom=2.54cm, top=2.54cm]{geometry}
\usepackage{amsmath}
\usepackage{graphicx}

\title{Bicycle movement}
\author{Juan Sebasti\'{a}n Mart\'{i}nez Ar\'{e}valo}

\begin{document}
\maketitle

\begin{abstract}
This document contains information about the project to model the movement of a bicycle. It briefly explains the theory of movement and the computational model used to obtain numerical results.
\end{abstract}

\section{First approach.}

The first model to consider is one in which the bicycle is frictionless. The equation that describe the movement of a body in classical mechanics is Newton's second law, which in one dimension is describes by:

\begin{equation}
  F = m\frac{dv}{dt}
\end{equation}

clearing $\frac{dv}{dt}$

\begin{equation}
  \frac{dv}{dt} = F/m
  \label{NSL}
\end{equation}

with this, we can determine the velocity of the bicycle at any time. However,  the force $F$ isn't easy to determine, so it's more convenient to perform the analysis in terms of other parameter like the power $P$, which represents the energy given to the physical system per unit of time. To do this, consider the mathematic definition of power, which is

\begin{equation}
  P = \frac{dE}{dt}
\end{equation}

since the energy in the system is purely kinetic, it takes the form $E=\frac{1}{2}mv^{2}$. Therefore the derivative is $dE/dt = mv(dv/dt)$. Clearing it

\begin{equation}
  \frac{dv}{dt} = \frac{P}{mv}
\end{equation}

integring, take the velocity at initial time $t_0$ as $v_0$ gives like result

\begin{equation}
  v(t) = \sqrt{v_0^2 + \frac{P}{m}t}
\end{equation}

However, this result is unphysical in the sense that the velocity increases indefinitely. Therefore, we consider the derivative of velocity as a finite difference


$$ \frac{v_{i+1} - v_i}{\Delta t} = \frac{P}{mv_{i}} \rightarrow v_{i+1} = v_i + \frac{P}{mv_{i}}\Delta t$$

Therefore, given an initial velocity, it is possible to obtain the velocity for any future time. A graph of velocity versus time is shown in the Figure \ref{f1}.

\begin{figure}[!h]
  \centering
  \includegraphics[width=0.6\linewidth]{velocity_v1.pdf}
  \caption{Velocity versus time in a model without friction. All parameters had taken as 1.0, with this respective units.}
  \label{f1}
\end{figure}

\section{With friction}

The problem with the model is that the velocity increases indefinitely. Obviously, in real life, the system does not behave like this. The reason for this is the lack of consideration of friction in the system which is very important for the final result. In the case of a bicycle, the friction with the ground and between the parts of system themselves doesn't have a significant impact on the final result, so we will ignore it and consider only the friction with air. To include friction in the equation, it is necessary to analyze the energy required for the cyclist to displace the mass of air. Doing this, as can found in the literature, results in adding a term to the velocity in the consideration of a finite difference. So

\begin{equation}
  v_{i+1} = v_i + \frac{P}{mv_{i}}\Delta t - \frac{C\rho A v_i^2}{2m}\Delta t
\end{equation}

with $C$ being a parameter that depends on the shape of the system, $\rho$ is the air density, and $A$ the frontal area of the system. With this consideration, the result is that the system now has a terminal velocity, as is shown in Figure \ref{F2}.

\begin{figure}[!h]
  \centering
  \includegraphics[width=0.6\linewidth]{velocity_v2.pdf}
  \caption{Velocity versus time in a model with friction. All parameters had taken as 1.0, with this respective units.}
  \label{F2}
\end{figure}

\end{document}
